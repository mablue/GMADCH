\documentclass[a4paper,12pt]{article}
\usepackage{amsmath,amssymb,graphicx,hyperref}
\usepackage{geometry}
\geometry{margin=1in}

\title{GMADCH: H-Design Modularization for Software Systems with Incoherent Call Graphs}
\author{
  Masoud Azizi\textsuperscript{1},
  Dr. Habib Izadkhah\textsuperscript{2},
  Prof. Ayaz Issazadeh\textsuperscript{3}\\
  \small{
  \textsuperscript{1}Department of Computer Science, Institute of Higher Education of Tabriz Scholars, Tabriz\\
  \textsuperscript{2}Department of Computer Science, Faculty of Mathematical Sciences, Tabriz University, Tabriz\\
  \textsuperscript{3}Department of Computer Science, Faculty of Mathematical Sciences, Tabriz University, Tabriz\\
  \texttt{mablue92@gmail.com}, \texttt{izadkhah@tabrizu.ac.ir}, \texttt{isazadeh@tabrizu.ac.ir}
  }
}
\date{October 2025}

\begin{document}
\maketitle

\begin{abstract}
Software benchmarking is critical for understanding system architecture, maintainability, and evolution. Scaling and modularizing software with incoherent call graphs is a nonlinear, hard problem. GMADCH introduces an H-design algorithm based on Levenshtein distance and vocabulary congruence, enabling logical grouping by conceptual similarity. This paper presents the mathematical foundation, algorithmic details, and empirical results, comparing GMADCH to prior graph-based and random approaches.
\end{abstract}

\section{Introduction}
Modern software systems often feature modules with weak or absent interconnections, resulting in incoherent call graphs. Existing graph-based modularization methods, such as those considering relationship depth (e.g., \cite{Izadkhah2016,Pourasghar2020}), struggle with such systems and may resort to random assignment. GMADCH replaces randomness with a text-based similarity approach, leveraging string metrics and vocabulary congruence for logical clustering.

\section{Definitions}
\subsection{Call Dependency Graph}
A call dependency graph $G = (V, E)$ represents entities (files, classes, functions) as nodes $V$ and their relationships (calls, references) as edges $E$.

\subsection{Levenshtein Distance}
Given two strings $s_1, s_2$, the Levenshtein distance $d_L(s_1, s_2)$ is the minimum number of edit operations (insertions, deletions, substitutions) required to transform $s_1$ into $s_2$ \cite{Levenshtein1966}. Formally,
\[
d_L(s_1, s_2) = 
\begin{cases}
|s_1| & \text{if } |s_2| = 0 \\
|s_2| & \text{if } |s_1| = 0 \\
\min \begin{cases}
d_L(\text{tail}(s_1), s_2) + 1 \\
d_L(s_1, \text{tail}(s_2)) + 1 \\
d_L(\text{tail}(s_1), \text{tail}(s_2)) + [s_1[0] \neq s_2[0]]
\end{cases}
\end{cases}
\]

\subsection{Vocabulary Congruence Scoring}
For each word $w$ in a file, its H-design score is:
\[
\text{score}(w) = \text{freq}(w) + \sum_{\substack{w' \in D\\w' \neq w}} \frac{\text{freq}(w')}{d_L(w, w')}
\]
where $D$ is the dictionary of words (either global or user-provided).

\section{GMADCH Algorithm}
\subsection{Preprocessing}
\begin{enumerate}
    \item Extract all words from code files, filtering out short words and programming keywords.
    \item Build the dictionary $D$ (global or user-specified).
\end{enumerate}

\subsection{Scoring and Tagging}
For each file:
\begin{enumerate}
    \item Calculate $\text{score}(w)$ for all words $w$ in the file.
    \item Select top $k$ tags (default $k=3$).
\end{enumerate}

\subsection{Clustering}
Group files in the same folder by shared tags. Conceptual clusters simplify maintenance and architecture analysis.

\section{Mathematical Foundations}
\subsection{Similarity Matrices}
Let $S_{ij}$ be the similarity between entities $i$ and $j$:
\[
S_{ij} = 1 - \frac{d_L(w_i, w_j)}{\max_{i,j} d_L(w_i, w_j)}
\]
The similarity matrix informs hierarchical clustering and modularization.

\subsection{Quality Metrics}
Following \cite{Izadkhah2016}, modularization quality $MQ$ is defined:
\[
MQ = \frac{i}{i + j}
\]
where $i$ is the number of internal edges, $j$ is the number of external edges.

\section{Empirical Results}
GMADCH was tested on open-source systems (e.g., Windows Calculator, trading platforms). Compared to previous graph-based and random algorithms, GMADCH achieved higher modularization quality (MoJo, MoJoFM), reduced clustering error, and improved maintainability.

\section{Discussion}
GMADCH generalizes to all major programming languages and supports both automatic and user-defined dictionaries. The H-design approach connects entities by conceptual similarity, avoiding random assignment and enabling logical scaling of incoherent call graphs.

\section{Conclusion}
GMADCH advances modularization for heterogeneous software, combining string metric theory and vocabulary analysis. Future work includes mapping to sparse graphs (e.g., Johnson's algorithm) and refining similarity matrices.

\section{References}
\bibliographystyle{plain}
\begin{thebibliography}{99}
\bibitem{Izadkhah2016} H. Izadkhah, I. Elgedawy and A. Isazadeh, "E-CDGM: An Evolutionary Call-Dependency Graph Modularization Approach for Software Systems," Cybernetics and Information Technologies, pp. 70-90, 2016.
\bibitem{Pourasghar2020} B. Pourasghar, H. Izadkhah, A. Isazadeh and S. Lotf, "A Graph-based Algorithm for Software Systems Modularization by Considering the Depth of Relationships," 2020.
\bibitem{Levenshtein1966} V.I. Levenshtein, "Binary codes capable of correcting deletions, insertions, and reversals," Soviet Physics Doklady, vol. 10, no. 8, pp. 707–710, 1966.
% Add further references from your resource list as needed
\end{thebibliography}

\end{document}
