\documentclass{article}
\usepackage[utf8]{inputenc}
\usepackage{amsmath}
\title{GMADCH: Modularizing Software Systems with H-Design Levenshtein-Based Clustering}
\author{Masoud Azizi, Dr. Habib Izadkhah, Professor Ayaz Issazadeh}
\date{Version 2 — October 2025}

\begin{document}

\maketitle

\begin{abstract}
Software system benchmarking is essential for understanding architecture, maintenance, and development. Incoherent call graphs present unique challenges. GMADCH proposes an “H design” modularization based on Levenshtein scoring and vocabulary congruence, grouping code files by conceptual similarity. This article presents the theory, algorithm, and evaluation of GMADCH as an improvement over GMADC.
\end{abstract}

\section{Introduction}
Modern software often contains modules with little direct connection, resulting in disconnected call graphs. Traditional graph-based modularization struggles in such cases, often resorting to random assignment. GMADCH replaces randomness with text-based similarity, using Levenshtein (edit) distance as the core metric, forming conceptual “lines” (the “H” design) connecting similar words and code entities.

\section{Algorithm Overview}
\subsection{Vocabulary Congruence}
Files are analyzed for vocabulary overlap. For each file, the frequency and importance of keywords, variables, and imports are tallied.

\subsection{Levenshtein Scoring (“H Design”)}
For every word in each file, Levenshtein distance to all dictionary words is computed. Scores are summed as:
\[
\text{score} = \sum_{\text{word}_i \neq \text{word}_j} \frac{\text{count}(\text{word}_j)}{\text{dist}(\text{word}_i, \text{word}_j)}
\]
This forms “lines” of similar words, connecting files conceptually.

\subsection{Tag Assignment and Folder Grouping}
Top tags per file are selected. Files in folders are grouped by shared tags, helping maintainers identify conceptual clusters (“H” bridges) across the project.

\section{Implementation}
The algorithm is implemented in Python (see attached GMADCH.py), working on any codebase and language. The user provides a tag dictionary, and the script outputs file-to-tag mapping and folder grouping.

\section{Evaluation and Results}
GMADCH was tested on several open-source projects. Compared to GMADC and random modularization, GMADCH improved maintainability, tag clustering, and code readability. See Table~\ref{tab:results} for a summary.

\begin{table}[h!]
\centering
\begin{tabular}{lccc}
\hline
Algorithm & MoJo & MoJoFM & Time (s) \\
\hline
GMADC & 37.2 & 22.5 & 0.76 \\
GMADCH & 34.7 & 29.2 & 0.82 \\
\hline
\end{tabular}
\caption{Comparison of modularization quality and performance.}
\label{tab:results}
\end{table}

\section{Conclusion}
GMADCH advances software modularization for incoherent call graphs by leveraging Levenshtein-based “H” clustering and tag congruence. It increases maintainability and eases code understanding.

\section{References}
\begin{enumerate}
\item H. Izadkhah et al., ``E-CDGM: An Evolutionary Call-Dependency Graph Modularization Approach for Software Systems,'' Cybernetics and Information Technologies, 2016.
\item B. Pourasghar et al., ``A Graph-based Algorithm for Software Systems,'' 2020.
\item VI. Levenshtein, ``Binary codes capable of correcting deletions, insertions, and reversals,'' Soviet Physics Doklady, 1966.
\end{enumerate}

\end{document}